\documentclass[11pt,a4paper]{article}
\usepackage[T1, T2A]{fontenc}
\usepackage[utf8]{inputenc}
\usepackage{amsthm}
\usepackage[english,russian]{babel} 
\usepackage{graphicx}
\usepackage[left=2cm,right=1cm,
top=2cm,bottom=3cm,bindingoffset=0cm]{geometry} 
\usepackage{hyperref}
\usepackage{amsmath}
\usepackage{amssymb}
\usepackage{amsthm}
\usepackage{mathtools}
\usepackage{hyperref}
\usepackage{float}
\usepackage{wrapfig}
\usepackage{indentfirst}
\usepackage{hyperref}
\usepackage[normalem]{ulem}
\usepackage{xcolor}
\usepackage{soul}
\newtheorem{lemma}{Лемма}
\newtheorem{definition}{Определение}
\newtheorem{properties}{Свойства}

\newtheorem{theorem}{Теорема}
\newtheorem{method}{Метод}

\theoremstyle{remark}
\newtheorem*{remark}{Замечание}
\newtheorem*{mem}{Напоминание}

% \newcommand*\backmatter{%
%  \setcounter{section}{0}%
%  \renewcommand\section{back.\arabic{section}}}


\DeclareMathOperator*{\argmin}{arg\,min}
\DeclareMathOperator*{\argmax}{arg\,max}
\DeclarePairedDelimiter{\norm}{\lVert}{\rVert}
\DeclareMathOperator\arctanh{arctanh}
\DeclareMathOperator{\sign}{sign}
\DeclareMathOperator{\const}{const}
\newcommand{\Cov}{\mathrm{Cov}}
\newcommand{\E}{\mathrm{E}}
\newcommand{\Var}{\mathrm{Var}}
\newcommand{\suml}{\sum\limits}
\newcommand\tab[1][1cm]{\hspace*{#1}}
\newcommand\given[1][]{\:#1\vert\:}
\newcommand\independent{\protect\mathpalette{\protect\independenT}{\perp}}
\def\independenT#1#2{\mathrel{\rlap{$#1#2$}\mkern2mu{#1#2}}}

\renewcommand{\le}{\leqslant}
\renewcommand{\ge}{\geqslant}
\renewcommand{\leq}{\leqslant}
\renewcommand{\geq}{\geqslant}
\newcommand{\eps}{\varepsilon}
\renewcommand{\phi}{\varphi}

\graphicspath{ {images/} }

% Цвета для гиперссылок
\definecolor{linkcolor}{HTML}{799B03} % цвет ссылок
\definecolor{urlcolor}{HTML}{799B03} % цвет гиперссылок

\hypersetup{pdfstartview=FitH,  linkcolor=linkcolor,urlcolor=urlcolor, colorlinks=true}

\title{Легковесный детектор для токсичного текстового контента}
\author{Михаил Мартинсон}


\begin{document}
	\maketitle
	
	\label{firstpage}
	
	\subsection*{Задача}
	
	В этом задании требовалось обучить лекгий и быстрый классификатор для короткий текстов. Датасет состоит из кротких сообщений людей\href{https://www.kaggle.com/c/jigsaw-toxic-comment-classification-challenge/overview}{Kaggle}.
	
	\subsection*{Модель}
	
	Так как в задании есть бонусная часть про интропретируемость модели, вначале я решил 
	
%	\addcontentsline{toc}{section}{Список используемой литературы}
%	
%	%далее сам список используемой литературы
%	\begin{thebibliography}{}
%		\bibitem{word_embed_review} 
%		\href{http://ad-publications.informatik.uni-freiburg.de/theses/Bachelor_Jon_Ezeiza_2017.pdf}{(en) A review of word
%			embedding and document similarity algorithms applied to academic text.} (про Vector Space Mode, Deep Learning, Word embeddings,  review of word embedding algorithms (Word2Vec, GloVe, FastText, WordRank), review of document similarity measures (Baseline: VSM, Doc2Vec, Doc2VecC, Word Mover’s Distance, Skip-thoughts, Sent2Vec)), 90 листов.
%		
%		\bibitem{review_exp_compare_text_cluster} 
%		\href{http://www.mathnet.ru/links/2a50b93bfe9152511e398d666cc65d1b/tisp214.pdf}{(ru) Обзор и экспериментальное сравнение методов кластеризации текстов.} Хороший материал для начальной навигации.
%		
%		\bibitem{HB1} 
%		\href{https://habr.com/post/324686/}{(ru) Технологический стек классификации текстов на естественных языках.} Описаны сигнатурный и шаблонный классификаторы, склейка в мультикласс.
%		
%		\bibitem{Stacking}
%		\href{http://www.machinelearning.ru/wiki/images/5/56/Guschin2015Stacking.pdf}{(ru) Методы ансамблирования обучающихся алгоритмов}
%		
%		
%		
%	\end{thebibliography}
			
		\end{document}